\documentclass[12pt]{report}
\usepackage[a4paper]{geometry}
\usepackage[myheadings]{fullpage}
\usepackage{fancyhdr}
\usepackage{lastpage}
\usepackage{graphicx, wrapfig, subcaption, setspace, booktabs}
\usepackage[T1]{fontenc}
\usepackage[font=small, labelfont=bf]{caption}
\usepackage{fourier}
\usepackage[protrusion=true, expansion=true]{microtype}
\usepackage[english]{babel}
\usepackage{sectsty}
\usepackage{url, lipsum}
\usepackage{float}
\usepackage{gensymb}
\usepackage{amsmath}
\graphicspath{ {./images/}}


\newcommand{\HRule}[1]{\rule{\linewidth}{#1}}
\onehalfspacing
\setcounter{tocdepth}{5}
\setcounter{secnumdepth}{5}

%-------------------------------------------------------------------------------
% HEADER & FOOTER
%-------------------------------------------------------------------------------
\pagestyle{fancy}
\fancyhf{}
\setlength\headheight{15pt}
\fancyhead[L]{Stephen Chen}
\fancyhead[R]{UBID: 50225769}
\fancyfoot[R]{Page \thepage\ of \pageref{LastPage}}
%-------------------------------------------------------------------------------
% TITLE PAGE
%-------------------------------------------------------------------------------

\begin{document}
\title{ \normalsize \textsc{ }
		\\ [2.0cm]
		\HRule{1pt} \\
		\LARGE \textbf{WMAP Attitude and Control}\\
        \normalsize \textbf{MAE425}\\
		\HRule{1pt} \\ [0.5cm]
		\normalsize  \vspace*{5\baselineskip}}



\author{Stephen Chen \\
		Student ID: 50225769 \\ 
		University at Buffalo \\
    }
        
\date {May 14, 2021}


\maketitle
\tableofcontents
\newpage

%-------------------------------------------------------------------------------
% Section title formatting
\sectionfont{\scshape}
%-------------------------------------------------------------------------------

%-------------------------------------------------------------------------------
% Abstract
%-------------------------------------------------------------------------------

\section*{Abstract}
\addcontentsline{toc}{section}{Abstract}

The purpose of this laboratory experiment is to use two different methods to calculate the drag coefficient of a basket-style coffee filter. The first method used will be a free-fall drop experiment where the only measurements taken will be the distance the filter drops for and the time it takes for it to fall that distance. By using an approximate calculation and an iterative process, the drag coefficient can be calculated. The second method uses a high speed camera to measure the filter's terminal speed which can then be used to calculate its drag coefficient. The results from both experiments did not match with each other nor the wind tunnel data although the result from the two experiments were fairly similar. The rationale for performing multiple methods of the same experiment is to ensure that the results from the experiments make sense.

%-------------------------------------------------------------------------------
% Introduction
%-------------------------------------------------------------------------------
\newpage
\section*{Introduction}
\addcontentsline{toc}{section}{Introduction}

In the previous laboratory experiments, we were able to measure and calculate the aerodynamic forces generated by various objects using a wind tunnel and force balance. Although this method works well to obtain empirical data, it is rather expensive and difficult to set up as it requires the use of specialized tools and setups. In this laboratory experiment, attempts will be made to analyze the drag force on a coffee filter using only a stop watch and a far drop. Another experiment will be performed using a high speed camera to measure the coffee filter's speed for comparison. These results will also be compared to wind tunnel data from the previous lab.\\

\begin{figure*}[h!]
	\centering
	% \fbox{\includegraphics[width = 1\textwidth]{cf.png}}
	\caption{Coffee Filter seen through high speed camera}
	\label{fig:cf}
\end{figure*}

\noindent The first experiment consists of dropping coffee filters of various mass from a pre-measured height and using a stopwatch to measure the amount of time the filters spend falling. The mass of the coffee filter can be changed by stacking multiple coffee filters together. By taking advantage of the filter's geometry, stacked filters do not change the total cross-sectional area by much but still adds mass to the total sum. As the cross-sectional area remains the same, the drag force created from the fall should remain the same as well between different masses.\\

\noindent To calculate the drag coefficient on the coffee filter using only a stopwatch, a free-body force balance of the coffee filter must be performed. Since we are assuming that the only force acting on the coffee filter is the body force or gravity and the drag force from the air, the force balance becomes a very simple equation of:

\begin{equation}
	\sum F = -W+D
\end{equation}
where $\sum F$ is the total force acting upon the coffee filter, $W$ is the force of gravity or weight of the coffee filter, and $D$ is the drag force from the air.\\

\noindent The weight of the coffee filter can be found with:

\begin{equation}
	W = mg
\end{equation}
where $m$ is the mass of the coffee filter and $g$ is the acceleration due to gravity which for this lab experiment we will approximate to be $g = 9.81 \frac{m}{s^2}$.\\

\noindent The drag force on an object can be calculated using:

\begin{equation}
	D = \frac{1}{2}\rho V^2 A C_D
\end{equation}
where $\rho$ is the air density, $V$ is the object's velocity, $A$ is the object's effective area, and $C_D$ is the drag coefficient. For this lab experiment, we will approximate the air density to be constant at $\rho = 1.225 \frac{kg}{m^3}$ due to the drop being short enough to assume that air density remains fairly constant.\\

\noindent At terminal velocity, the coffee filter falls at a constant rate. Therefore, its acceleration is zero. Using Newton's second law, we know that the net force on the object at terminal velocity must be zero.

\begin{equation}
	\sum F = ma = 0
\end{equation}
where $a$ is the object's acceleration.\\

\noindent Combining Equations 1, 2, 3, and 4 we get:
\begin{equation}
	mg = \frac{1}{2}\rho V^2 A C_D
\end{equation}
\begin{equation}
	V^2 = \frac{2g}{\rho A C_D} m
\end{equation}

\noindent We can see from the above equation that it can be equated to a standard linear function of the form:
\begin{equation}
	y = mx + b
\end{equation}
where y corresponds to the velocity squared, m corresponds to the term $\frac{2g}{\rho A C_D}$ which includes the drag coefficient, and x corresponds to the mass. b in this case is equal to 0 but could be different depending on the data collected.\\

\noindent Using the similarity between Equation 6 and the standard linear function, it means that by plotting the velocity squared term and the mass on a graph, the slope can be used to calculate the drag coefficient. However, the important thing to note is that this is only true at terminal velocity as it is the only instance the drag force and weight are equal. Despite this, by assuming the velocity was at terminal velocity still allows for the approximation of the drag coefficient.\\

\begin{equation}
	C_D = \frac{2g}{(slope)\rho A}
\end{equation}

\noindent The way the coffee filter falls can be separated into two phases. For the first phase, the fall of the coffee filter is unsteady. This means that the velocity of the coffee filter is increasing from zero, the instance the coffee filter is dropped, to the terminal velocity. During this unsteady phase, the force of gravity and the drag force is not equal, therefore the method mentioned before cannot be used. During the second phase, the object has reached terminal velocity and will have constant velocity for the rest of the fall, allowing us to use the previously mentioned method. To more accurately calculate the drag coefficient, the unsteady time, the time the coffee filter spent in unsteady fall, must be subtracted from the total fall time. Then, the unsteady fall distance must also be subtracted from the total height the filter falls for. What is left will then only be the steady phase of the fall which we can then use to calculate the drag coefficient.\\

\noindent To calculate the unsteady time, a function of velocity in terms of time is required. Using Equations 1, 2, 3, and 4 once again but this time not setting the acceleration to zero, we obtain the following equation. The acceleration of an object can be shown as the derivative of the object's velocity with respect to time or:
\begin{equation}
	-mg + \frac{1}{2}\rho C_D v(t)^2 A = ma = m\frac{dv}{dt}
\end{equation}

\noindent By rearranging terms, we get:
\begin{equation}
	v'(t) = \frac{1}{2m}\rho C_D A v(t)^2 - g
\end{equation}
where $v'(t)$ is another way of representing $\frac{dv}{dt}$. What results is a differential equation that can be solved to obtain a function of velocity in terms of time. The derivation of the function is shown in Appendix A.\\

\begin{equation}
	v(t) = -\frac{\sqrt{g} tanh(\sqrt{\frac{1}{2m}\rho C_D A g}t)}{\sqrt{\frac{1}{2m}\rho C_D A}}
\end{equation}

\noindent By calculating the limit of the velocity function while time approaches infinity, we can calculate the terminal velocity. The hyperbolic tangent function approaches 1 as the input approaches infinity so the terminal velocity is:

\begin{equation}
	V_{term} = -\frac{\sqrt{g}}{\sqrt{\frac{1}{2m}\rho C_D A}}
\end{equation}

\noindent Comparing the terminal velocity expression to the velocity function, we can see that to obtain the time required to reach $99\%$ of the terminal velocity,
\begin{equation}
	tanh(\sqrt{\frac{1}{2m}\rho C_D A g}t_{us}) = 0.99
\end{equation}
where $t_{us}$ is the unsteady time or time required to reach $99\%$ of the terminal velocity.\\

\noindent This equation can then be solved for $t_{us}$,
\begin{equation}
	t_{us}=\frac{2.64665}{\sqrt{\frac{1}{2m}\rho C_D A g}}
\end{equation}

\noindent To calculate how far the filter has traveled during the unsteady phase of the fall, the velocity equation can be integrated with respect to time to obtain the displacement function in terms of time. A derivation of the equation is shown in Appendix B.

\begin{equation}
	d(t) = -\frac{ln(cosh(\sqrt{\frac{1}{2m}\rho C_D A g}t))}{\sqrt{\frac{1}{2m}\rho C_D A}}
\end{equation}

\noindent We can see from Equation 14 that in order to find the unsteady time, we require the drag coefficient. This problem can be resolved by using an iterative process to "guess" the drag coefficient. By using the guess drag coefficient and going through the calculation process to calculate the steady velocity squared, a new drag coefficient can be found and the process repeated until the difference between the new and old values are smaller than a preset error value. By using the Solver Add-in in Microsoft Excel, this process can be performed many times automatically (See Appendix C).\\

\noindent The second experiment will be performed using a high speed camera, a drop tube, and a grid backdrop. Coffee filters will be dropped at some height and recorded right before touching the ground. After falling for this distance, the velocity of the coffee filter will be assumed to be at terminal velocity, allowing us to calculate the drag coefficient using information from the footage of the high speed camera. \\ 

\noindent To obtain the velocity from the high speed camera, the grid backdrop will be used to measure the distance traveled. To calculate the time taken to travel this distance, counting frames and the frame-rate of the video can be used. An important thing to note while performing this experiment was that the perspective of the camera can affect the data collected. Because of this, the distance between where the filter lands and the camera or the backdrop must be measured along with the distance between the camera and the backdrop. By using these measurements and trigonometry, the actual fall distance of the filter can be calculated.\\

%-------------------------------------------------------------------------------
% Methods
%-------------------------------------------------------------------------------
\newpage
\section*{Methods}
\addcontentsline{toc}{section}{Methods}

The experiment was split into two parts. The first part of the experiment will focus on the long drop and the timing of the drop using a stopwatch. The second part of the experiment will use the high speed camera to record the drop of the coffee filters.\\

\noindent To perform the first part of the experiment, the following steps were taken:

\begin{enumerate}
	\item Using a ruler, the average diameter of the coffee filter was measured. \textbf{NOTE:} the diameter measured is of the unflattened coffee filter and averaged over 4 different filters.
	\item Using an analytical scale, the average mass of a coffee filter was found. \textbf{NOTE:} the average mass was found by measuring the total mass of several filters and dividing by the number of filters measured.
	\item The total drop distance of the filter was measured using a tape measure to measure the approximate distance between the release point and the ground.
	\item Starting with a single coffee filter, the filter was dropped and the drop time was recorded. If the drop was incomplete or affected by external factors, the trial was invalid. Figure \ref{fig:filterdrop} shows the view of the filter dropping.
	\begin{figure}[h!]
		\centering
		% \includegraphics[width = .7\textwidth]{filterdrop.png}
		\caption{Timing filter drops}
		\label{fig:filterdrop}
	\end{figure}
	\item After waiting 20 seconds after the drop for the air to settle, another trial was done.
	\item Drops of the same mass was completed 3 times.
	\item Adding another filter to the stack, the experiment was repeated until data for up to 10 filters were collected. 
\end{enumerate}

\noindent To perform the second part of the experiment, the following steps were taken:

\begin{enumerate}
	\item The high speed camera was set up at the bottom of the drop tunnel, facing the grid backdrop.
	\item The cooling box for the camera was placed over the camera while making sure the camera still looked out through the cutout in the box walls.
	\item By using a round level, the camera was checked to make sure that it is level.
	\item The cooling box was turned on.
	\item The camera was connected to the computer using a USB cable and turned on.
	\item The camera software "Ueye Cockpit" was run in Advanced Mode with Optimal Colors.
	\item The green triangle "Play" symbol was clicked and the camera was opened. The camera's view is now shown on the computer. 
	\item The settings of the camera was adjusted in the software to the following.
	\begin{enumerate}
		\item Size and profile was set to: 640 x 480
		\item Pixel clock and frame rate set to highest possible value
		\item Frame rate was set to "hold"
		\item Exposure time was set to 1 ms and "hold"
	\end{enumerate}
	\item A coffee filter was placed where it is expected to fall and camera was focused on the filter.
	\item Using the drop tunnel setup and vacuum, a filter was attached to the vacuum hose.
	\item The vacuum hose was returned to the top of the drop tunnel while still holding on to the filter(s). Setup should resemble Figure \ref{fig:vacuumfilter} below.
	\begin{figure}[h!]
		\centering
		% \includegraphics[width = .6\textwidth]{vacuumfilter.png}
		\caption{Coffee filters held in position with vacuum}
		\label{fig:vacuumfilter}
	\end{figure}
	\item The door to the drop tunnel was closed and the air was allowed to settle for 30 seconds.
	\item Using the film reel button, a new recording was started in the camera software and the Max Frame Rate setting was set to the maximum allowed value.
	\item The recording was started and the vacuum was turned off to drop the coffee filter(s).
	\item After the filter(s) has reached the ground, the recording was stopped. The software was exited and the camera was disconnected after each recording to prevent overheating.
	\item The distance between the center of the coffee filter and the camera was recorded.
	\item The recording was checked to ensure that the coffee filter lands within view of the camera. 
	\item Steps 10-18 were repeated with 4 times for each number of filters up to 4 filters, resulting in 16 recordings.
\end{enumerate}

\noindent To analyze the data of the first part of the experiment, the following steps were taken:

\begin{enumerate}
	\item The assumed steady-state velocity was calculated from the total drop height and the time taken for the filter to complete the drop.
	\item The square of this steady-state velocity was plotted against the mass on a graph.
	\item Using the slope of this plot, the steady-state drag coefficient was calculated using Equation 8.
	\item Using Equation 14 and the steady-state drag coefficient as a guess, the unsteady time was found. 
	\item Using unsteady time and Equation 15, unsteady displacement was found.
	\item Subtracting the unsteady time from the drop time and unsteady displacement from the total height, the steady time and steady height was found.
	\item The velocity is calculated by dividing the steady height by the steady time and squared.
	\item The velocity squared is plotted against mass and slope was used to recalculate drag coefficient.
	\item Steps 4-8 was repeated automatically using Excel's Solver Add-in with the total squared difference between the steady times as error. (See Appendix C)
	\item After final drag coefficient was found, simulation of the drop using MATLAB's ode45 function was run using the found drag coefficient.
\end{enumerate}

\noindent To analyze the high speed camera part of the experiment, the following steps were taken:
\begin{enumerate}
	\item The videos were analyzed using Media Player Classic by scrubbing through frame by frame. 
	\item By using a corner of the filter and the grid backdrop, the distance traveled was recorded in terms of frames.
	\item The frame rate was taken from the properties of the video file.
	\item Using trigonometry and the distance measurements between the camera, filter, and the backdrop, the actual distance traveled was calculated. 
	\item Using the frame rate and frame count from Step 2, the travel time was recorded.
	\item Using the distance and time, the velocity was calculated and squared.
	\item Plotting the velocity squared and the mass, the slope was used to calculate the drag coefficient.
\end{enumerate}

%-------------------------------------------------------------------------------
% Results and Discussion
%-------------------------------------------------------------------------------
\newpage
\section*{Results and Discussion}
\addcontentsline{toc}{section}{Results and Discussion}

The results of the experiments were shown below.

\subsection*{Presentation of Results}
\addcontentsline{toc}{subsection}{Presentation of Results}

The steady-state approximation of the drag coefficient was found to be
\begin{equation*}
	C_{D,ss} = 0.8192
\end{equation*}

\noindent The uncertainty of the steady-state drag coefficient was found to be
\begin{equation*}
	\delta C_{D,ss} = 0.0135
\end{equation*}

\noindent Using Excel and Solver, the corrected drag coefficient taking unsteady time and distance into account was found to be
\begin{equation*}
	C_{D,us} = 0.8715
\end{equation*}	

\noindent The uncertainty of the unsteady drag coefficient was found to be
\begin{equation*}
	\delta C_{D,us} = 0.0158
\end{equation*}

\noindent The high speed camera experimental drag coefficient was found to be
\begin{equation*}
	C_{D,HS} = 0.6229
\end{equation*}

\noindent The uncertainty of the high speed camera drag uncertainty was found to be
\begin{equation*}
	\delta C_{D,HS} = 0.1243
\end{equation*}

\noindent Using MATLAB and ode45 to simulate the filter drop using the corrected drag coefficient, the following plots were generated.

\begin{figure}
	\centering
	% \fbox{\includegraphics[width = .75\textwidth]{filterdropsim.png}}
	\caption{Simulated Filter Drops for 1 and 10 Filters}
	\label{fig:filterdropsim}
\end{figure}

\begin{figure}
	\centering
	% \fbox{\includegraphics[width = .75\textwidth]{filterdropsim1.png}}
	\caption{Simulated Filter Drop vs. Experiment for 1 Filter}
	\label{fig:filterdropsim1}
\end{figure}

\begin{figure}
	\centering
	% \fbox{\includegraphics[width = .75\textwidth]{filterdropsim10.png}}
	\caption{Simulated Filter Drop vs. Experiment for 10 Filters}
	\label{fig:filterdropsim10}
\end{figure}

\begin{figure}
	\centering
	% \fbox{\includegraphics[width = .75\textwidth]{filterdropsimvel.png}}
	\caption{Simulated Filter Drop Velocity for 1 and 10 Filters}
	\label{fig:filterdropsimvel}
\end{figure}

\begin{figure}
	\centering
	% \fbox{\includegraphics[width = .75\textwidth]{filterdropsimvel1.png}}
	\caption{Simulated Filter Drop Velocity vs. Excel for 1 Filter}
	\label{fig:filterdropsimvel1}
\end{figure}

\begin{figure}
	\centering
	% \fbox{\includegraphics[width = .75\textwidth]{filterdropsimvel10.png}}
	\caption{Simulated Filter Drop Velocity vs. Excel for 10 Filters}
	\label{fig:filterdropsimvel10}
\end{figure}

\newpage
\subsection*{Discussion}
\addcontentsline{toc}{subsection}{Discussion}

1. The uncertainty of the steady-state drag coefficient was found to be $\delta C_{D,ss} = 0.0153$. With this uncertainty, the corrected drag coefficient is not within the steady-state drag coefficient analysis. Assuming that the corrected drag coefficient is valid, the percent error of the steady-state analysis drag coefficient was found to be:

\begin{equation*}
	\%error = \frac{|0.8192-0.8715|}{0.8715}*100\% = 6.00\%
\end{equation*}


\noindent 2. We use 99\% of the terminal speed as a cut-off because 100\% of the terminal speed requires infinite time. We can see from Equations 11 and 12 that to obtain 100\% terminal velocity, 
\begin{equation}
	tanh(\sqrt{\frac{1}{2m}\rho C_D A g}t_{us}) = 1
\end{equation}
\noindent It is only possible for the hyperbolic tangent function to reach 1 when the input to the function is infinity, therefore, it would require infinite time to reach terminal velocity.\\

\noindent 3. From Figures 4-6, we can see that the simulated filter drop is very close to the experimental filter drop. The marked points on the figures indicate the time at which the filter hit the ground in the physical experiment. We can see that the time at which the simulated filter crosses the height = 0 line is very close to the experimental time for both 1 filter and 10 filters. Although the 1 filter simulation was much closer to the real experiment, both had time differences of less than 0.5 seconds.\\

\noindent 4. Figures 7-9 show the velocity of the simulated filter drop for 1 and 10 filters with the marked points indicating the values the Excel worksheet calculates that the filter has hit 99\% terminal velocity. We can see that the simulated filter drop matches very well to the Excel Solver-corrected data, with both points lying on the line.\\

\noindent 5. From the different method of analysis performed, we can see that the results from the free-fall time trials did not match with the velocity measurements using the high speed camera. Neither the free-fall filter drop results nor the high speed camera results match that of the wind tunnel where $C_D = 1.78$. The high speed camera drag coefficient was also subject to very high uncertainty of close to 20\% which could be enough to invalidate the data. Although it's not clear why the data from the high speed camera experiment was so different from the free-fall drop experiment and simulation, possible explanations could be that the frame rate of the recording were too low, the video analysis was not precise enough, or the camera perspective skewed the data by a significant amount. Since the free-fall experiment only required a few measurements, height, time, and diameter, there are not as many areas where uncertainty can build up. The high speed camera experiment required many measurements including frame rate, distance between camera and backdrop, distance between filter and backdrop, distance traveled on the grid from the perspective of the camera and multiple calculations that can propagate them. These could be possible reasons why the uncertainty is so high and why the data did not match up with the free-fall experiment.

\subsection*{Uncertainty Analysis}
\addcontentsline{toc}{subsection}{Uncertainty Analysis}

To calculate the uncertainty of the steady state drag coefficient. We can use Equation 8 and substitute in formula for area.

\begin{equation}
	C_D = \frac{8g}{(slope)\rho \pi D^2}
\end{equation}

\noindent Since the only values that have uncertainty are the slope and the diameter, the formula for the drag coefficient uncertainty becomes:

\begin{equation}
	\delta C_D = \sqrt{(\frac{\partial C_D}{\partial slope})^2(\delta slope)^2+(\frac{\partial C_D}{\partial D})^2(\delta D)^2}
\end{equation}

\noindent The partial derivatives were calculated to be:
\begin{equation}
	\frac{\partial C_D}{\partial slope} = -\frac{8g}{(slope)^2\rho \pi D^2}
\end{equation}
\begin{equation}
	\frac{\partial C_D}{\partial D} = -\frac{16g}{(slope)\rho \pi D^3}
\end{equation}

\noindent The diameter uncertainty is the smallest measurement on the ruler which is $\delta D = 0.001 m$. The slope uncertainty was given by linest function on Excel to be $\delta slope = 10.667$.

\noindent Plugging in values, 
\begin{equation}
	\delta C_D = \sqrt{(-\frac{8g}{(slope)^2\rho \pi D^2})^2(\delta slope)^2+(-\frac{16g}{(slope)\rho \pi D^3})^2(\delta D)^2}
\end{equation}
\begin{equation*}
	\delta C_{D,ss} = \sqrt{(-\frac{8(9.81)}{(1022.854)^2(1.225) \pi (0.156)^2})^2(10.667)^2+(-\frac{16(9.81)}{(1022.854)(1.225) \pi (0.156)^3})^2(0.001)^2}
\end{equation*}
\begin{equation*}
	\delta C_{D,ss} = 0.0135
\end{equation*}

\noindent To calculate the uncertainty of the unsteady drag coefficient, the same equation can be used.

\begin{equation*}
	\delta C_{D,us} = \sqrt{(-\frac{8(9.81)}{(961.567)^2(1.225) \pi (0.156)^2})^2(10.379)^2+(-\frac{16(9.81)}{(961.567)(1.225) \pi (0.156)^3})^2(0.001)^2}
\end{equation*}
\begin{equation*}
	\delta C_{D,us} = 0.0158
\end{equation*}

\noindent To calculate the uncertainty of the high speed camera experiment, the equations used to calculate the drag coefficient must be combined:

\begin{equation}
	D =\frac{L'-S}{L'}D'
\end{equation}
where D is the distance traveled by the coffee filter, L' is the distance from the camera to the backdrop, S is the distance from the coffee filter to the backdrop, and D' is the distance measured from the camera that the filter traveled.

\begin{equation}
	V =\frac{L'-S}{L't}D'
\end{equation}
where t is time the filter took to travel distance D.

\begin{equation}
	slope = \frac{V^2}{m} = \frac{(L'-S)^2D'^2}{L'^2t^2m}
\end{equation}
where m is the mass of the filter.\\

\noindent Then, to calculate the uncertainty of the slope,
\begin{multline}
	\delta slope = \bigg[\bigg(\frac{\partial slope}{\partial L'}\bigg)^2(\delta L')^2+\bigg(\frac{\partial slope}{\partial D'}\bigg)^2(\delta D')^2+...\\
	...+\bigg(\frac{\partial slope}{\partial S}\bigg)^2(\delta S)^2+\bigg(\frac{\partial slope}{\partial t}\bigg)^2(\delta t)^2+\bigg(\frac{\partial slope}{\partial m}\bigg)^2(\delta m)^2\bigg]^{0.5}
\end{multline}

\noindent Calculating the partial derivatives,

\begin{equation}
	\frac{\partial slope}{\partial L'}=\frac{2SD'^2(L'-S)}{t^2L'^3m}
\end{equation}
\begin{equation}
	\frac{\partial slope}{\partial S}=\frac{2D'^2(S-L')}{L'^2t^2m}
\end{equation}
\begin{equation}
	\frac{\partial slope}{\partial D'}=\frac{2(L'-S)^2D'}{L'^2t^2m}
\end{equation}
\begin{equation}
	\frac{\partial slope}{\partial t}=-\frac{2(L'-S)^2D'^2}{L'^2t^3m}
\end{equation}
\begin{equation}
	\frac{\partial slope}{\partial m}=-\frac{(L'-S)^2D'^2}{L'^2t^2m^2}
\end{equation}

\noindent The measurement uncertainties are $\delta L' = 0.001 m$, $\delta D' = 0.0127 m$, $\delta S = 0.001 m$, $\delta t = \frac{1}{fps}$, and $\delta m = 0.0001 g = 10^{-7} kg$. fps represents the frame rate at which the video is captured at.

\noindent Then, by using Equations 25-30, we can calculate the uncertainty of the slope. Then, with that uncertainty, we can once again use Equation 21 to calculate the drag coefficient uncertainty. (See Appendix D)

\noindent Doing so for each data point of the high speed camera experiment, the average drag coefficient uncertainty was found to be:
\begin{equation*}
	\delta C_{D,HS} = 0.1243
\end{equation*}

%-------------------------------------------------------------------------------
%Conclusion
%-------------------------------------------------------------------------------
\newpage
\section*{Conclusion}
\addcontentsline{toc}{section}{Conclusion}

The free-fall filter drop experiment allows for the calculation of the drag coefficient of a coffee filter using only a large fall distance and a stopwatch through an iterative process. Simulation ran using the results of the drag coefficient calculation resulted in data that very closely matches results from the physical experiment. The high speed camera experiment returned results that did not match with the free-fall filter drop experiment. There is also very large uncertainties that result from the high speed camera data possibly due to the experiment procedure and instruments used. Neither of the results of the experiments match with the data from the wind tunnel. The takeaway of this laboratory experiment is that the process of experimentation must be understood completely as many different parts must come together to achieve meaningful results. But through that process, uncertainties can add up and propagate very quickly so as experimenters, it is important to know and predict what could happen.

%-------------------------------------------------------------------------------
%References
%-------------------------------------------------------------------------------
%\newpage
\section*{References}
\addcontentsline{toc}{section}{References}

\noindent [1] NASA GRC (2020). The Drag Coeffcient. Retrieved from: https://www.grc.nasa.gov/www/k-
\indent 12/airplane/dragco.html\\

\noindent [2] University at Buffalo (Fall 2020). Unsteady Drag Corrections on a Falling Coffee Filter Lab 
\indent Manual. Retrieved from: https://learn-us-east-1-prod-fleet02-xythos.content.blackboard
\indent cdn.com/5e00ea752296c/12263551


\newpage
\section*{Appendix}
\addcontentsline{toc}{section}{Appendix}

\subsection*{Appendix A}
\addcontentsline{toc}{subsection}{Appendix A}
\noindent Solving the differential equation
\begin{equation*}
	v'(t) = \frac{1}{2m}\rho C_D A v(t)^2 - g
\end{equation*}
\noindent Substituting variables to make solving simpler,
\begin{equation*}
	a = \frac{1}{2m}\rho C_D A
\end{equation*}
\begin{equation*}
	b = g
\end{equation*}
\noindent Then, the equation becomes
\begin{equation*}
	v'(t) -av(t)^2+b = 0
\end{equation*}
\noindent Then, to solve,
\begin{equation*}
	\frac{\frac{dv(t)}{dt}}{-b+av(t)^2}=1
\end{equation*}
\begin{equation*}
	\int\frac{\frac{dv(t)}{dt}}{-b+av(t)^2}dt=\int1dt
\end{equation*}
\begin{equation*}
	-\frac{tanh^{-1}(\frac{\sqrt{a}v(t)}{\sqrt{b}})}{\sqrt{a}\sqrt{b}}=t+c_1
\end{equation*}
\begin{equation*}
	v(t) = -\frac{\sqrt{b}tanh(\sqrt{a}\sqrt{b}(t+c_1))}{\sqrt{a}}
\end{equation*}
\noindent Using initial condition $v(0) = 0$,
\begin{equation*}
	v(0) = -\frac{\sqrt{b}tanh(\sqrt{a}\sqrt{b}c_1)}{\sqrt{a}} = 0
\end{equation*}
\begin{equation*}
	c_1 = 0
\end{equation*}
\noindent Then, 
\begin{equation*}
	v(t) = -\frac{\sqrt{b}tanh(\sqrt{a}\sqrt{b}t)}{\sqrt{a}}
\end{equation*}
\begin{equation*}
	v(t) = -\frac{\sqrt{g} tanh(\sqrt{\frac{1}{2m}\rho C_D A g}t)}{\sqrt{\frac{1}{2m}\rho C_D A}}
\end{equation*}

\subsection*{Appendix B}
\addcontentsline{toc}{subsection}{Appendix B}
\noindent Integrating the equation with respect to time
\begin{equation*}
	\int v(t)dt = \int -\frac{\sqrt{g} tanh(\sqrt{\frac{1}{2m}\rho C_D A g}t)}{\sqrt{\frac{1}{2m}\rho C_D A}}dt
\end{equation*}
\begin{equation*}
	\int v(t)dt = \int -\frac{\sqrt{b}tanh(\sqrt{a}\sqrt{b}t)}{\sqrt{a}}dt
\end{equation*}
\begin{equation*}
	u = \sqrt{a}\sqrt{b}t
\end{equation*}
\begin{equation*}
	\frac{du}{dt} = \sqrt{a}\sqrt{b}
\end{equation*}
\begin{equation*}
	dt = \frac{1}{\sqrt{a}\sqrt{b}}du
\end{equation*}
\begin{equation*}
	= -\frac{1}{a}\int tanh(u)du
\end{equation*}
\noindent Solving for $\int tanh(u)du$, \textbf{NOTE}: v-sub not the same variable as v velocity.
\begin{equation*}
	=\int \frac{sinh(u)}{cosh(u)}du
\end{equation*}
\begin{equation*}
	v= cosh(u)
\end{equation*}
\begin{equation*}
	\frac{dv}{du}=sinh(u)
\end{equation*}
\begin{equation*}
	du = \frac{1}{sinh(u)}dv
\end{equation*}
\begin{equation*}
	=\int \frac{1}{v}dv
\end{equation*}
\begin{equation*}
	=ln(v)
\end{equation*}
\noindent Substituting back in u, v,
\begin{equation*}
	d(t) = -\frac{ln(cosh(\sqrt{a}\sqrt{b}t))}{a}+C
\end{equation*}
\noindent Substituting back in a, b,
\begin{equation*}
	d(t) = -\frac{ln(cosh(\sqrt{\frac{1}{2m}\rho C_D A g}t))}{\sqrt{\frac{1}{2m}\rho C_D A}}
\end{equation*}

\newpage
\subsection*{Appendix C}
\addcontentsline{toc}{subsection}{Appendix C}
\noindent Using Solver to calculate corrected drag coefficient

\begin{figure*}[h!]
	\centering
	% \fbox{\includegraphics[width = .9\textwidth]{excelsolver.png}}
	\label{fig:excelsolver}
\end{figure*}
\begin{figure*}[h!]
	\centering
	% \fbox{\includegraphics[width = .6\textwidth]{excelsolver2.png}}
	\label{fig:excelsolver2}
\end{figure*}

\newpage
\subsection*{Appendix D}
\addcontentsline{toc}{subsection}{Appendix D}
\noindent Calculating uncertainty of high speed camera data

\begin{figure*}[h!]
	\centering
	% \fbox{\includegraphics[width = 1\textwidth]{hsdata.png}}
	\label{fig:hsdata}
\end{figure*}


\end{document}
